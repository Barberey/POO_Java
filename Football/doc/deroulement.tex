\newpage
\section{Déroulement du projet}
\label{sec:deroulement}

Dans cette section, nous décrivons comment le projet a été réalisé en équipe : la répartition des tâches, la synchronisation du travail en membres de l'équipe, etc.

\subsection{Répartition des taches et organisation globale}

\paragraph{}
    Dans un premier temps, nous détaillerons la répartition des différentes taches au sein du groupe et leur espacement dans le temps tel quelle fut imaginer.

\begin{center}
\begin{tabular}{ |c|c|c|c|c|c| } 
 \hline
   & Boyan & Romain & Guillaume & temps estimé & temps réel\\ 
 \hline\hline
   Cahier des charges & V & V & V & 4 jours &1 semaine\\ 
 \hline
  Conception graphique & V & V & V & 2 jours & 2 jours\\ 
 \hline
 Déplacement de la balle & X & X & V & 1 semaine & 2 semaine\\ 
 \hline
 Gestion des actions & V & V & X & 2 semaines &2 semaines\\
 \hline
 Réalisation des actions & V & V & X & 2 semaines & 3 semaines\\
 \hline
 Classe de donnée & V & V & V & 5 jours & 4 jours\\
 \hline
 IHM & V & X & X & Tout le long & Tous le long\\
 \hline
  Chart & X & V & V & 2 jours & 2 jours\\
 \hline
   CNP & V & V & V & 4 jours & 5 jours\\
 \hline
    Présentation technique & V & V & V & 5 jours & 5 jours\\
 \hline
 Rapport & V & V & V & 3 jours & 3 jours\\  
 \hline
\end{tabular}
\end{center}

\paragraph{}
    Dans un second temps, nous comparerons l'estimation à la réalité du projet.
    
\paragraph{}
    La différence de temps peut être une source de frustration dans tout projet. Cependant, la solution pour y remédier est simple : il suffit de faire preuve de réalisme et de prudence dès le début. En effet, nous avons reconnu que notre ambition excessive et notre manque de prévoyance nous ont empêchés de faire face efficacement aux difficultés imprévues, qui ont considérablement ralenti notre progression. Pour éviter cela, il est crucial de prendre en compte les risques potentiels dès le début du projet et de planifier des marges de manœuvre en termes de temps pour les imprévus. En outre, nous devons adopter une approche plus prudente quant à nos attentes et objectifs, en veillant à ce qu'ils soient réalistes et réalisables dans les délais impartis. En adoptant cette approche, nous pourrons minimiser les risques et les retards, tout en garantissant la qualité et la réussite de notre projet.
    
\subsection{Défis rencontrés}

\paragraph{}
    Ici, nous décrirons les différentes problématiques rencontrée tel que le fait de programmer à 3 sur un projet, etc.

\paragraph{}
    Parmi les nombreux défi rencontrés tout au long de la réalisation du projet, on retrouve principalement la collaboration à trois sur un projet aussi minutieux. En effet, il n’a pas toujours été évident de se repartir les tâches ou encore de programmer chacun de son coté sur le même projet simultanément.
\paragraph{}
    De plus, la gestion du temps fait également l’objet d’un défi conséquent dans l’aboutissement du projet. Il est clair que nous avons souvent eu à se re concentrer sur le projet de façon assidu afin de ne pas prendre trop de retard, car le semestre est passé très vite et il est bien plus simple de prendre du retard que d'en rattraper.
\paragraph{}
    Également, de gros défis algorithmique ont été rencontrés et surmontés. Une simulation de jeu de foot est évidemment pas mince affaire. Ainsi, nous avons tous eu a se creuser la tête un grands nombres de fois afin de surmonter ces défis et réaliser un jeu de foot réaliste et parfaitement fonctionnel.
\paragraph{}
    En fin de compte, notre plus grand défi a été de créer un projet qui reflète notre vision et notre identité, tout en répondant aux attentes et aux besoins de notre projet. Nous avons travaillé dur pour concevoir un produit qui nous plaît et dont nous sommes fiers, tout en étant convaincus qu'il soit un minimum convaincant. Cela n'a pas été facile, car il y avait des compromis à faire et des choix à faire tout au long du processus de développement. Cependant, nous avons pris soin d'écouter les commentaires et les retours de nos proches pour prendre des décisions éclairées et garantir la qualité de notre projet. Nous sommes convaincus que nous avons réussi à créer un produit qui répond à nos exigences et qui rempli une bonne parti des consignes.
\newpage
\section{Conclusion et perspectives}
\label{sec:conclusion}

\noindent Dans cette section, nous résumerons la réalisation du projet et nous présenterons également les extensions et améliorations possibles du projet de façons objectives et qualitatives. 

\subsection{Achèvement du projet}

\paragraph{}
    Dans cette sous-section, nous détaillerons le niveau d'achèvement de notre projet, par rapport aux attentes de bases, et aux objectifs qui nous avions été fixées au préalable dans le cahier des charges (\ref{sec:specification}).

    Le fait que le projet soit à 80\% terminé est une bonne nouvelle, car cela signifie que la majeure partie du travail a déjà été effectuée. Bien qu'il reste quelques petits détails à régler, comme l'ajout de l'endurance des joueurs et des pouvoirs inclus dans le cahier des charges, il est important de reconnaître que ce retard a permis d'améliorer certaines parties du projet. En effet, en prenant le temps de peaufiner les détails déjà en place, l'équipe a pu affiner les fonctionnalités existantes pour les rendre plus performantes et plus efficaces. De plus, cela a également permis de mieux comprendre les besoins des utilisateurs et de s'adapter en conséquence. Ainsi, même si le projet n'est pas encore complètement terminé, cette situation a en fait permis d'optimiser son potentiel et de répondre aux attentes des utilisateurs de manière plus satisfaisante.
    

\subsection{Points d'améliorations possibles}

\paragraph{}
    Dans cette sous-section, nous exprimerons les possibles points d'améliorations de notre projet, au vu de notre réalisation et en lien avec notre niveau d'achèvement.

\paragraph{}
    Tout d'abord, on pense que l'un des points faibles de notre projet est le manque de super pouvoirs. Contrairement à ce qu'on avait initialement prévu, on a décidé de ne pas inclure de super pouvoirs dans le jeu, ce qui pourrait limiter la variété de jeu et de stratégies que les joueurs peuvent utiliser. En conséquence, on va travailler sur des alternatives pour permettre aux joueurs d'explorer différents styles de jeu et de stratégies sans l'utilisation de super pouvoirs.

De plus, on est conscient que le manque de choix stratégique dans le jeu peut également être un point faible. Bien qu'on ait ajouté quelques options de stratégie telles que les changements de formation et les ajustements tactiques, on reconnaît que cela pourrait ne pas être suffisant pour les joueurs qui cherchent à personnaliser leur jeu. Pour améliorer cela, on va travailler sur l'ajout de plus d'options de stratégie, telles que les substitutions et les remplacements pour permettre aux joueurs de s'adapter rapidement à la situation sur le terrain. Cependant, un autre aspect lié à l'absence de choix stratégique que nous avons identifié est le manque de cartons jaunes et rouges dans le jeu. Nous allons travailler sur l'ajout de ces éléments pour une expérience de jeu plus réaliste et immersive.

Au surplus, un autre point faible que nous avons identifié est le manque de cartons jaunes et rouges dans le jeu. Les sanctions pour les fautes sont une partie importante du football, et sans elles, le jeu peut sembler moins réaliste et moins engagé. Nous allons travailler sur l'ajout de cartons jaunes et rouges pour les fautes commises par les joueurs, afin de mieux refléter les règles et les normes du football.

En outre, nous avons également réalisé que les remplaçants ne servent pas à grand-chose dans notre projet. Bien qu'ils soient inclus dans le jeu, leur utilité est limitée car ils ne peuvent pas être utilisés pour influencer directement le jeu. Pour résoudre ce problème, nous allons travailler sur l'ajout de plus d'options de remplacement et de substitutions pour permettre aux joueurs de mieux gérer leur équipe et de s'adapter à la situation sur le terrain. Nous pensons que cela permettra aux joueurs de mieux utiliser les remplaçants pour influencer le jeu et ainsi améliorer l'expérience globale de jeu.

Un autre point faible de notre projet est le manque de style et de réalisme dans le jeu. Bien qu'on ait ajouté quelques éléments visuels et sonores pour renforcer l'ambiance du jeu, on reconnaît que cela pourrait ne pas être suffisant pour créer une expérience de jeu véritablement immersive. Pour améliorer cela, on va travailler sur l'ajout de plus d'options de personnalisation pour les joueurs, les équipes et les stades. On va également travailler sur la physique du ballon et des joueurs pour rendre le jeu plus réaliste. Nous avons également remarqué que les remplaçants ne servent actuellement à rien dans le jeu, nous allons donc travailler sur l'ajout de fonctionnalités pour permettre aux joueurs de mieux gérer leur banc de remplaçants.

Enfin, on reconnaît que le manque de fluidité dans le jeu peut être un point faible. On est conscient que les ralentissements et les latences peuvent affecter l'expérience de jeu des joueurs. Pour résoudre ce problème, on va travailler sur l'optimisation du jeu en utilisant des serveurs dédiés et en offrant des options de personnalisation pour permettre aux joueurs d'optimiser les performances selon leur matériel.

En conclusion, bien qu'on ait créé un projet de simulation de football intéressant, il y a plusieurs points que l'on reconnaît comme étant des points faibles. Cependant, on est déterminé à améliorer ces points faibles en ajoutant plus de super pouvoirs, d'options de stratégie, de personnalisation et d'optimisation pour garantir une expérience de jeu plus immersive et agréable pour les joueurs. Nous sommes conscients que cela nécessite un travail conséquent, mais nous sommes persuadés que cela améliorera considérablement la qualité de notre jeu.
